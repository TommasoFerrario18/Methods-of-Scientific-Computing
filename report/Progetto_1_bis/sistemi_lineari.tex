\chapter{Progetto 1 bis: Mini libreria per sistemi lineari}
\section{Introduzione}
Per la realizzazione della libreria contenente i metodi per la risoluzione di
sistemi lineari, è stato scelto di utilizzare \href{https://julialang.org/}{\textbf{Julia}},
un linguaggio di programmazione open-source, sviluppato per ottenere prestazioni
elevate e con una sintassi simile a quella di Python e MATLAB. Essendo concepito
per la manipolazione efficace del calcolo scientifico, offre una vasta gamma di
librerie per la gestione di matrici e vettori.

In particolare, per lo sviluppo di questo progetto sono state utilizzate due
librerie della Standard Library di Julia:
\begin{itemize}
    \item \textbf{LinearAlgebra}: fornisce funzioni per la manipolazione di
          matrici e vettori.
    \item \textbf{SparseArrays}: fornisce funzioni per la gestione di matrici sparse.
\end{itemize}

L'impiego di quest'ultima libreria è stato fondamentale per ridurre l'occupazione
di memoria, in quanto le matrici utilizzate negli esperimenti sono matrici sparse.
Nello specifico, sono state utilizzate le seguenti matrici sparse:
\begin{itemize}
    \item \textbf{spa1}: matrice di dimensione 1000x1000 con 182,434 elementi non nulli.
    \item \textbf{spa2}: matrice di dimensione 3000x3000 con 1,633,298 elementi non nulli.
    \item \textbf{vem1}: matrice di dimensione 1681x1681 con 13,385 elementi non nulli.
    \item \textbf{vem2}: matrice di dimensione 2601x2601 con 21,225 elementi non nulli.
\end{itemize}

Per permettere la riproducibilità degli esperimenti, vogliamo riportare di seguito
le caratteristiche del sistema utilizzato per la realizzazione della libreria e
per l'esecuzione degli esperimenti. Tutti gli esperimenti sono stati eseguiti su
un computer con le seguenti caratteristiche:
\begin{itemize}
    \item CPU: Intel Core i5-1135G7
    \item RAM: 16 GB
    \item Sistema Operativo: Windows 11
    \item Julia: versione 1.10.2
\end{itemize}
\section{Struttura della libreria}
La libreria realizzata è composta da tre moduli principali:
\begin{itemize}
    \item \textbf{IterativeMethods}: contiene i metodi iterativi per la risoluzione
          di sistemi lineari.
    \item \textbf{DirectMethods}: contiene i metodi diretti per la risoluzione
          di sistemi lineari.
    \item \textbf{Utils}: contiene le funzioni di utilità per la manipolazione
          di matrici e vettori.
\end{itemize}

\subsection{Utils}
Nel modulo \textbf{Utils} sono presenti le funzioni per svolgere compiti di
utilità. Tra queste troviamo:
\begin{itemize}
    \item \textbf{read\_sparse\_matrix}: funzione per la lettura di una matrice
          da file in formato \textbf{.mtx}. Tale funzione restituisce una matrice
          sparsa.
    \item \textbf{check\_sizes}: funzione per il controllo delle dimensioni di
          una matrice e del vettore dei termini noti.
\end{itemize}
\subsection{DirectMethods}
Nel modulo \textbf{DirectMethods} è presente il metodo che effettua la risoluzione
di un sistema lineare in cui la matrice è triangolare inferiore, tramite il
metodo di sostituzione in avanti.

Di seguito riportiamo lo pseudocodice del metodo \textbf{forward\_substitution}:
\begin{verbatim}
function forward_substitution(A, b)
    n = size(A, 1)
    x = zeros(n)
    x[1] = b[1] / A[1, 1]
    for i in 2
        x[i] = (b[i] - dot(A[i, 1], x[1])) / A[i, i]
    end
    return x
end
\end{verbatim}
Questo metodo è stato implementato poiché nel metodo di risoluzione di Gauß-Seidel,
per cui la regola di aggiornamento è la seguente:
\begin{equation}
    x^{(k+1)} = x^{(k)} + P^{-1}(b - Ax^{(k)})
\end{equation}
con $P$ è una matrice triangolare inferiore, prevede il calcolo di una matrice
inversa. Dato che tale operazione è computazionalmente costosa, possiamo evitare
di calcolare la matrice inversa e al suo posto risolvere il seguente sistema
lineare:
\begin{equation}
    Py = b - Ax^{(k)}
\end{equation}
dove $y$ è il vettore che otteniamo risolvendo il sistema lineare con il metodo
della sostituzione in avanti. In questo modo, possiamo calcolare la soluzione
$x^{(k+1)}$ come:
\begin{equation}
    x^{(k+1)} = x^{(k)} + y
\end{equation}
\subsection{IterativeMethods}
Nel modulo \textbf{IterativeMethods} sono presenti i metodi iterativi per la
risoluzione di sistemi lineari. In particolare, sono stati implementati i seguenti
metodi:
\begin{itemize}
    \item \textbf{Jacobi}: metodo di Jacobi per la risoluzione di sistemi lineari.
    \item \textbf{GaussSeidel}: metodo di Gauß-Seidel per la risoluzione di
          sistemi lineari.
    \item \textbf{Gradient}: metodo del gradiente per la risoluzione di sistemi
          lineari.
    \item \textbf{ConjugateGradient}: metodo del gradiente coniugato per la
          risoluzione di sistemi lineari.
\end{itemize}

Tutti i metodi implementati utilizzano come criterio di arresto il confronto del
residuo riscalato, calcolato come:
\begin{equation}
    \frac{\|b - Ax^{(k)}\|}{\|b\|}
\end{equation}
dove $x^{(k)}$ è la soluzione al passo $k$ e $b$ è il vettore dei termini noti.
La tolleranza per questo confronto è un valore fornito dall'utente.

Inoltre, per evitare cicli infiniti, è possibile specificare un numero massimo
di iterazioni. Il valore predefinito è fissato a 20000 iterazioni.

La libreria è stata progettata a partire dalla definizione di una funzione generica
\textbf{GenericIterativeMethod}, la quale prende in input la matrice $A$, il
vettore dei termini noti $b$, la tolleranza, il numero massimo di iterazioni, e
il metodo di aggiornamento della soluzione. Questa funzione implementa una logica
comune a tutti i metodi iterativi, come rappresentato dal seguente pseudocodice:
\begin{verbatim}
function GenericIterativeMethod(A, b, tol, max_iter, update_method)
    n = size(A, 1)
    x = zeros(n)
    r = b - A * x
    res = norm(r) / norm(b)
    iter = 0
    while res > tol
        x = update_method(A, b, x)
        r = b - A * x
        res = norm(r) / norm(b)
        iter += 1
        if iter == max_iter
            print("Numero massimo di iterazioni raggiunto")
            return x
        end
    end
    return x
end
\end{verbatim}

Per ciascun metodo iterativo richiesto, è stato definito un metodo specifico che
implementa la regola di aggiornamento della soluzione. Questo metodo restituisce
la soluzione aggiornata al passo $k+1$ e, dato che il criterio di arresto è basato
sul residuo riscalato, tutti i metodi richiedono il calcolo del residuo. Pertanto,
è stato deciso di restituire anche il residuo calcolato, per evitare di ricalcolarlo
ogni volta. La logica di aggiornamento della soluzione sfrutta le funzioni per
la manipolazione di matrici e vettori fornite dalla libreria \textbf{LinearAlgebra}.

In aggiunta, la libreria contiene metodi di interfaccia che richiamano direttamente
la funzione generica, passando il metodo di aggiornamento corrispondente al metodo
iterativo richiesto.

Di seguito sono presentate le implementazioni dei metodi iterativi richiesti.

\subsubsection{Metodo di Jacobi}
Per il metodo di Jacobi, la regola di aggiornamento della soluzione è la seguente:
\begin{equation}
    x^{(k+1)} = x^{(k)} + P^{-1}(b - Ax^{(k)})
\end{equation}
In questo caso, il calcolo della matrice inversa $P^{-1}$ è efficiente poiché $P$
è una matrice diagonale; di conseguenza, basta calcolare il reciproco degli
elementi sulla diagonale.

Durante la definizione di questo metodo, è stata prevista l'implementazione dei
metodi \textbf{JOR} e \textbf{Richardson}, introducendo i parametri $\omega$ e
$\alpha$, con i relativi controlli per verificarne la presenza e gli intervalli
di valori ammissibili.

\subsubsection{Metodo di Gauß-Seidel}
Per il metodo di Gauß-Seidel, la regola di aggiornamento della soluzione è la seguente:
\begin{equation}
    x^{(k+1)} = x^{(k)} + y
\end{equation}
dove $y$ è il vettore ottenuto risolvendo il sistema lineare $Py = b - Ax^{(k)}$.
Questo è possibile poiché la matrice $P$ è una matrice triangolare inferiore,
quindi possiamo utilizzare il metodo della sostituzione in avanti per risolvere
il sistema lineare, evitando il calcolo della matrice inversa.

Anche per questo metodo è stata prevista l'implementazione dei metodi \textbf{SOR}
e \textbf{Richardson}, introducendo i parametri $\omega$ e $\alpha$, con i relativi
controlli per verificarne la presenza e gli intervalli di valori ammissibili.

\subsubsection{Metodo del Gradiente}
Per il metodo del gradiente, la regola di aggiornamento della soluzione è la seguente:
\begin{equation}
    x^{(k+1)} = x^{(k)} + \alpha r^{(k)}
\end{equation}
dove $r^{(k)}$ è il residuo al passo $k$ e $\alpha$ è il fattore di scala calcolato
come:
\begin{equation}
    \alpha = \frac{\langle r^{(k)}, r^{(k)}\rangle}{\langle r^{(k)}, Ar^{(k)}\rangle}
\end{equation}

\subsubsection{Metodo del Gradiente Coniugato}
Questo metodo rappresenta una versione migliorata del metodo del gradiente, in
cui si evita il problema della convergenza a "zig-zag". La regola di aggiornamento
della soluzione è la seguente:
\begin{equation}
    x^{(k+1)} = x^{(k)} + \alpha d^{(k)}
\end{equation}
dove $\alpha$ è il fattore di scala calcolato come:
\begin{equation}
    \alpha = \frac{\langle d^{(k)}, r^{(k)}\rangle}{\langle d^{(k)}, Ad^{(k)}\rangle}
\end{equation}
e $d^{(k)}$ è la direzione di discesa al passo $k$, calcolata ad ogni iterazione
con la formula:
\begin{equation}
    d^{(k)} = r^{(k)} + \beta_{k-1} d^{(k-1)}
\end{equation}
dove $\beta$ è il fattore di scala calcolato come:
\begin{equation}
    \beta_k = \frac{\langle d^{(k)T}, (Ar^{(k + 1)})\rangle}{\langle d^{(k)T}, (Ad^{(k)})\rangle}
\end{equation}
con $\langle \cdot, \cdot \rangle$ è il prodotto scalare tra due vettori.

\section{Risultati sperimentali}
Implementati i metodi, si è proceduto con la valutazione delle prestazioni di
ciascuno di essi. In particolare, sono stati eseguiti degli esperimenti sulle
matrici precedentemente descritte, utilizzando come vettore dei termini noti il
risultato della moltiplicazione tra la matrice e un vettore di 1 di dimensione
pari al numero di righe della matrice. Oltre a questo, i metodi sono stati testati
al variare della tolleranza, utilizzando i valori $10^{-4}$, $10^{-6}$, $10^{-8}$
e $10^{-10}$.

Per quanto riguarda le misurazioni dei tempi di esecuzione e l'occupazione di
memoria, sono state utilizzate la funzione \textbf{time}, invocata prima e dopo
l'esecuzione del codice, e la macro \textbf{@allocated} per misurare l'occupazione
di memoria. Quest'ultima restituisce la quantità di memoria allocata in byte.
