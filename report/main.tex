\documentclass[a4paper, oneside]{report}
\usepackage[italian]{babel}
\usepackage[utf8]{inputenc}
\usepackage[a4paper,top=2.5cm,bottom=2.5cm,left=2cm,right=2cm]{geometry}
\usepackage[table,xcdraw]{xcolor}
\usepackage{booktabs}
\usepackage{pdfpages}
\usepackage{pgfplots}
\usepackage{fancyhdr}
\usepackage{caption}
\usepackage{subcaption}
\usepackage{hyperref}
\usepackage{amsmath}
\usepackage{amssymb}
\usepackage{soul}

\usepackage{algorithm}
\usepackage{algpseudocode}

\pagestyle{fancy}
\fancyhead[L,RO]{\slshape \rightmark}
\fancyfoot[C]{\thepage}

\title{Progetto Calcolo scientifico}
\author{
    Telemaco Terzi (865981) (\href{https://github.com/Tezze2001}{@Tezze2001}) \\\\
    Tommaso Ferrario (869005) (\href{https://github.com/TommasoFerrario18}{@TommasoFerrario18})
    }
\date{\today}

\pgfplotsset{compat=1.13}

\begin{document}

\maketitle
\newtheorem{teorema}{Teorema}
\newtheorem{dimostrazione}{Dimostrazione}
\newtheorem{definizione}{Definizione}
\newtheorem{esempio}{Esempio}
\newtheorem{nota}{Nota}

\tableofcontents

\chapter{Progetto 1 bis: Mini libreria per sistemi lineari}
\section{Introduzione}
Per la realizzazione della libreria contenente i metodi per la risoluzione di
sistemi lineari, è stato scelto di utilizzare \href{https://julialang.org/}{\textbf{Julia}},
un linguaggio di programmazione open-source, sviluppato per ottenere prestazioni
elevate e con una sintassi simile a quella di Python e MATLAB. Essendo concepito
per la manipolazione efficace del calcolo scientifico, offre una vasta gamma di
librerie per la gestione di matrici e vettori.

In particolare, per lo sviluppo di questo progetto sono state utilizzate due
librerie della Standard Library di Julia:
\begin{itemize}
    \item \textbf{LinearAlgebra}: fornisce funzioni per la manipolazione di
          matrici e vettori.
    \item \textbf{SparseArrays}: fornisce funzioni per la gestione di matrici sparse.
\end{itemize}

L'impiego di quest'ultima libreria è stato fondamentale per ridurre l'occupazione
di memoria, in quanto le matrici utilizzate negli esperimenti sono matrici sparse.
Nello specifico, sono state utilizzate le seguenti matrici sparse:
\begin{itemize}
    \item \textbf{spa1}: matrice di dimensione 1000x1000 con 182,434 elementi non nulli.
    \item \textbf{spa2}: matrice di dimensione 3000x3000 con 1,633,298 elementi non nulli.
    \item \textbf{vem1}: matrice di dimensione 1681x1681 con 13,385 elementi non nulli.
    \item \textbf{vem2}: matrice di dimensione 2601x2601 con 21,225 elementi non nulli.
\end{itemize}

Per permettere la riproducibilità degli esperimenti, vogliamo riportare di seguito
le caratteristiche del sistema utilizzato per la realizzazione della libreria e
per l'esecuzione degli esperimenti. Tutti gli esperimenti sono stati eseguiti su
un computer con le seguenti caratteristiche:
\begin{itemize}
    \item CPU: Intel Core i5-1135G7
    \item RAM: 16 GB
    \item Sistema Operativo: Windows 11
    \item Julia: versione 1.10.2
\end{itemize}
\section{Struttura della libreria}
La libreria realizzata è composta da tre moduli principali:
\begin{itemize}
    \item \textbf{IterativeMethods}: contiene i metodi iterativi per la risoluzione
          di sistemi lineari.
    \item \textbf{DirectMethods}: contiene i metodi diretti per la risoluzione
          di sistemi lineari.
    \item \textbf{Utils}: contiene le funzioni di utilità per la manipolazione
          di matrici e vettori.
\end{itemize}

\subsection{Utils}
Nel modulo \textbf{Utils} sono presenti le funzioni per svolgere compiti di
utilità. Tra queste troviamo:
\begin{itemize}
    \item \textbf{read\_sparse\_matrix}: funzione per la lettura di una matrice
          da file in formato \textbf{.mtx}. Tale funzione restituisce una matrice
          sparsa.
    \item \textbf{check\_sizes}: funzione per il controllo delle dimensioni di
          una matrice e del vettore dei termini noti.
\end{itemize}
\subsection{DirectMethods}
Nel modulo \textbf{DirectMethods} è presente il metodo che effettua la risoluzione
di un sistema lineare in cui la matrice è triangolare inferiore, tramite il
metodo di sostituzione in avanti.

Di seguito riportiamo lo pseudocodice del metodo \textbf{forward\_substitution}:
\begin{verbatim}
function forward_substitution(A, b)
    n = size(A, 1)
    x = zeros(n)
    x[1] = b[1] / A[1, 1]
    for i in 2
        x[i] = (b[i] - dot(A[i, 1], x[1])) / A[i, i]
    end
    return x
end
\end{verbatim}
Questo metodo è stato implementato poiché nel metodo di risoluzione di Gauß-Seidel,
per cui la regola di aggiornamento è la seguente:
\begin{equation}
    x^{(k+1)} = x^{(k)} + P^{-1}(b - Ax^{(k)})
\end{equation}
con $P$ è una matrice triangolare inferiore, prevede il calcolo di una matrice
inversa. Dato che tale operazione è computazionalmente costosa, possiamo evitare
di calcolare la matrice inversa e al suo posto risolvere il seguente sistema
lineare:
\begin{equation}
    Py = b - Ax^{(k)}
\end{equation}
dove $y$ è il vettore che otteniamo risolvendo il sistema lineare con il metodo
della sostituzione in avanti. In questo modo, possiamo calcolare la soluzione
$x^{(k+1)}$ come:
\begin{equation}
    x^{(k+1)} = x^{(k)} + y
\end{equation} 

\subsection{IterativeMethods}
Nel modulo \textbf{IterativeMethods} sono presenti i metodi iterativi per la
risoluzione di sistemi lineari. In particolare, sono stati implementati i seguenti
metodi:
\begin{itemize}
    \item \textbf{Jacobi}: metodo di Jacobi per la risoluzione di sistemi lineari.
    \item \textbf{GaussSeidel}: metodo di Gauß-Seidel per la risoluzione di
          sistemi lineari.
    \item \textbf{Gradient}: metodo del gradiente per la risoluzione di sistemi
          lineari.
    \item \textbf{ConjugateGradient}: metodo del gradiente coniugato per la
          risoluzione di sistemi lineari.
\end{itemize}

Tutti i metodi implementati utilizzano come criterio di arresto il confronto del
residuo riscalato, calcolato come:
\begin{equation}
    \frac{\|b - Ax^{(k)}\|}{\|b\|}
\end{equation}
dove $x^{(k)}$ è la soluzione al passo $k$ e $b$ è il vettore dei termini noti.
La tolleranza per questo confronto è un valore fornito dall'utente.

Inoltre, per evitare cicli infiniti, è possibile specificare un numero massimo
di iterazioni. Il valore predefinito è fissato a 20000 iterazioni.

\section{Risultati sperimentali}
Implementati i metodi, si è proceduto con la valutazione delle prestazioni di
ciascuno di essi. In particolare, sono stati eseguiti degli esperimenti sulle
matrici precedentemente descritte, utilizzando come vettore dei termini noti il
risultato della moltiplicazione tra la matrice e un vettore di 1 di dimensione
pari al numero di righe della matrice. Oltre a questo, i metodi sono stati testati
al variare della tolleranza, utilizzando i valori $10^{-4}$, $10^{-6}$, $10^{-8}$
e $10^{-10}$.

Per quanto riguarda le misurazioni dei tempi di esecuzione e l'occupazione di
memoria, sono state utilizzate la funzione \textbf{time}, invocata prima e dopo
l'esecuzione del codice, e la macro \textbf{@allocated} per misurare l'occupazione
di memoria. Quest'ultima restituisce la quantità di memoria allocata in byte.


\subsection{Metodo di Jacobi}
\subsection{Metodo di Gauß-Seidel}
\subsection{Metodo del gradiente}
\subsection{Metodo del gradiente coniugato}
\chapter{Progetto 2: Compressione di immagini tramite la DCT}

\section{Introduzione}
Per la realizzazione della compressione di immagini, è stato scelto di utilizzare
\href{https://julialang.org/}{\textbf{Julia}}, per i medesimi motivi del progetto
precedente.

In questo progetto è stata implementata l'equazione della discrete cosine transform $2$ (DCT2).
La sua implementazione è stata realizzata applicando sequenzialmente sulla matrice
la discrete cosine transform $1$ (DCT) prima per righe e poi
per colonne. In questo modo, è stata aumentata la riutilizzabilità del codice
e ridotto il rischio di introdurre errori.

In particolare per lo sviluppo sono state utilizzate le librerie:
\begin{itemize}
    \item \textbf{LinearAlgebra}
    \item \textbf{FileIO}: fornisce funzioni per la lettura e scrittura di file.
    \item \textbf{Images}: fornisce funzioni per la trattazione di immagini.
    \item \textbf{\href{https://github.com/JuliaMath/FFTW.jl}{FTTW}}: fornisce
          funzioni che implementano la FTT.
\end{itemize}

Tutte le librerie sono open source, più precisamente FTTW implementa un binding
con la libreria di C.

\section{DCT2 custom}
Per prima cosa è stata implementata DCT1,
la cui sua formula è stata riportata nell'equazione \ref{eq:dct}.

\begin{equation}
    X_k = \begin{cases}
        \sqrt{\frac{1}{N}}\cdot \sum_{n=0}^{N-1} x_n \cos\left[\frac{\pi}{N}\left(n + \frac{1}{2}\right) n \right] & k=1     \\
        \sqrt{\frac{2}{N}}\cdot\sum _{n=0}^{N-1}x_{n}\cos \left[\frac{\pi}{N}\left(n+\frac{1}{2}\right)n\right]    & k \ne 1
    \end{cases}
    \label{eq:dct}
\end{equation}

Per l'implementazione, si è pensato di utilizzare un approccio basato su operazioni
tra vettori e matrici, più precisamente si pre-calcola la matrice
della base dei coseni (per righe) prima di effettuare il calcolo dei coefficienti.
Successivamente si effettua un prodotto matrice-vettore per ottenere il vettore
dei coefficienti, infine si normalizza il vettore dei coefficienti.

Dato un generico vettore $X_k$, per applicare la DCT si effettua:
\begin{equation*}
    dct(X_k) = norm(B_{\cos}\cdot X_k )
\end{equation*}

dove $\cdot$ rappresenta il prodotto matrice-vettore, $B_{\cos}$
è la base dei coseni creata per riga e $X_k$ è il vettore colonna iniziale.
I coefficienti così ottenuti vengono normalizzati nel seguente modo:
\begin{itemize}
    \item Il primo coefficiente viene moltiplicato per $\sqrt{\frac{1}{N}}$
    \item I restanti vengono moltiplicati per $\sqrt{\frac{2}{N}}$
\end{itemize}

L'implementazione della DCT2 si articola nel seguente modo:
\begin{itemize}
    \item si genera la base dei coseni per riga
    \item si applica inplace la DCT1 per righe
    \item si applica inplace la DCT1 per colonne
\end{itemize}

Utilizzando questa strategia si riesce ad ottenere dei tempi di esecuzione
comparabili alla DCT2 implementata dalla libreria usando la FFT.

Il codice della DCT è all'interno del file \textbf{Dct2.jl}, mentre il codice dei
test della dct e il codice per generare il grafico di complessità è nella cartella
\textbf{test\_dct.jl}.

\subsection{Studio di complessità}

La complessità della DCT2 custom viene analizzata per passi nei seguenti punti:
\begin{itemize}
    \item \textbf{generazione della base dei coseni}: questa operazione richiede
          un numero costante di operazioni tra scalari e un calcolo del coseno per ogni
          entry della matrice. Avremo quindi un totale $\mathcal{O}(N^2)$ per generare
          la matrice, quando $N\times N$ è la dimensione della matrice (dimensione dello
          spazio vettoriale per numero di coefficienti).
    \item \textbf{applicazione della DCT1 su un vettore}: richiede di effettuare
          un prodotto matrice-vettore e successivamente di normalizzare i coefficienti.
          Il prodotto matrice-vettore ha una complessità asintotica di
          $\mathcal{O}(N^2)$, dove $N$ è la dimensione del vettore. Mentre per
          normalizzare i coefficienti si richiede una scansione
          lineare del vettore che richiede $\mathcal{O}(N)$, dove $N$ è la dimensione
          del vettore. Quindi complessivamente si ha una complessità di $\mathcal{O}(N^2 + N) = \mathcal{O}(N^2)$.
    \item \textbf{applicazione della DCT2 sull'intera matrice}: l'applicazione della DCT2
          sull'intera matrice richiede di eseguire una DCT1 per ogni riga e poi per ogni
          colonna. All'atto pratico, non si rigenera ogni volta la matrice dei coseni,
          bensì inizialmente la si pre-calcola ($\mathcal{O}(N^2)$), successivamente si
          esegue la DCT1 per righe che richiede $\mathcal{O}(N \cdot N^2)$ ($N$ righe
          per $N^2$ la moltiplicazione riga-matrice) e successivamente si applica la
          DCT1 per colonne che richiede $\mathcal{O}(N \cdot N^2)$ ($N$ colonne
          per $N^2$ la moltiplicazione riga-matrice). Complessivamente si ottiene
          un $\mathcal{O}(N^3)$ a livello asintotico.
\end{itemize}

L'implementazione custom della DCT2 è stata confronta anche con quella ottimizzata
dalla libreria usando la FFT e nell'immagine \ref{fig:analisi_complex} si
può vedere il confronto.

Per generare il grafico è stata utilizzata la libreria standard \textbf{Plot} ed
è stata utilizzata la macro \textbf{@elapsed} per ottenere i tempi di esecuzione
delle funzioni.

Tutti i test sono stati realizzati su una macchina con:
\begin{itemize}
    \item Sistema Operativo: Windows $11$
    \item CPU: Ryzen $7$ $4800H$
    \item RAM: $16GB$
    \item Julia: versione 1.10.2
\end{itemize}

\newpage
\begin{figure}[!ht]
    \centering
    \includegraphics[width=0.5\textwidth]{Progetto_2/img/times_plot.png}
    \caption{Grafico dei tempi al variare della dimensione della matrice. Le linee
        continue rappresentano i dati empirici, le linee tratteggiate rappresentano
        l'andamento teorico dei metodi.}
    \label{fig:analisi_complex}
\end{figure}

Come si può notare dal grafico, l'andamento empirico segue quello teorico, ovvero
la DCT custom segue $\mathcal{O}(N^3)$, mentre la DCT della libreria segue $\mathcal{O}(N^2\log N)$.

In aggiunta, parlando dello spazio occupato, la DCT custom necessita di una matrice
$N \times N$ della base dei coseni e alloca un array di dimensione $N$ per salvare
i coefficienti, quindi si ottiene una complessità spaziale di $\mathcal{O}(N^2)$.

Si può ridurre la complessità spaziale effettuando tutte le operazioni inplace.

\section{Compressione}
Per facilitare l'uso del software che comprime l'immagine in scala di grigi, si
è pensato di sviluppare un'interfaccia mediante l'utilizzo di un framework open
source chiamato \href{https://github.com/plotly/dash}{Dash}. Il quale è un binding
in Julia della sua controparte in Python, generalmente utilizzato per data
visualizzation.

\subsection{Implementazione dell'interfaccia}
L'interfaccia consiste in una pagina web che chiede in input l'immagine da comprimere
e i parametri di compressione ($f$ e $d$), in output restituisce il l'immagine compressa.

Nell'interfaccia vengono mostrate entrambe le immagini, sia quella di input sia quella
compressa, in modo da confrontare la qualità delle immagini.

Il sorgente dell'interfaccia è contenuto all'interno del file \textbf{gui.jl}.

Una volta che il server web è in esecuzione, per accedere all'interfaccia basta
digitare sul browser il seguente link: \href{http://localhost:8050}{http://localhost:8050}.

\subsection{Implementazione della compressione}
Dopo l'implementazione della GUI che si occupa di prendere in input l'immagine e
i due parametri, è stato sviluppato l'algoritmo di compressione dell'immagine.
L'algoritmo è all'interno del file \textbf{Dct2.jl} sotto la funzione \textbf{ApplyDct2OnImage}.

L'implementazione della compressione si articola nei seguenti passi:
\begin{itemize}
    \item \textbf{ridimensionamento dell'immagine in input}: l'immagine è stata
          ridimensionata in modo tale da avere altezza e larghezza multiple del
          parametro $f$.
    \item \textbf{implementazione della compressione}: la compressione è stata
          implementata applicando sui sotto-quadrati $f\times f$ dell'immagine
          la DCT2 della libreria, successivamente sono state azzerate le entry
          $M_{k,l}$ tali che $k+l\ge d$, infine si applica IDCT2 della libreria e
          si normalizzano i valori.
    \item \textbf{visualizzazione output}: infine viene visualizzato l'output
          messo a confronto con l'input.
\end{itemize}

Nel dettaglio, la \textbf{ridimensionamento dell'immagine in input} elimina le ultime
righe e le ultime colonne dell'immagine in input in modo da avere le dimensioni
multiple di $f$.

Mentre, l'\textbf{implementazione della compressione} si articola nell'allocazione
di una matrice di appoggio $C\in \mathbb{R}^{f\times f}$ che viene inizializzata
con una sotto-matrice $f\times f$ dell'immagine che si vuole elaborare. Successivamente
si applica la DCT2 della libreria, si azzerano i coefficienti $C_{kl}$ tali che
$k+l\ge d$, in seguito si applica la IDCT2 della libreria, infine si arrotondano
all'intero più vicino i coefficienti e si riportano i valori superiori a $255$
e inferiori a $0$ rispettivamente ai valori $255$ e $0$. Infine, si copia la
sotto-matrice di appoggio nell'immagine. Questo passo viene ripetuto in tutta
l'immagine suddividendola in blocchi $f \times f$ disgiunti.

\begin{figure}[!ht]
    \centering
    \begin{subfigure}[!ht]{0.45\textwidth}
        \includegraphics[width=\textwidth]{Progetto_2/img/gui.png}
        \caption{Immagine originale.}
    \end{subfigure}
    \begin{subfigure}[!ht]{0.45\textwidth}
        \includegraphics[width=\textwidth]{Progetto_2/img/gui_compressed1.png}
        \caption{Immagine compressa F = 8 e d = 1.}
    \end{subfigure}
    \begin{subfigure}[!ht]{0.45\textwidth}
        \includegraphics[width=\textwidth]{Progetto_2/img/gui_compressed.png}
        \caption{Immagine compressa F = 8 e d = 4.}
    \end{subfigure}
    \begin{subfigure}[!ht]{0.45\textwidth}
        \includegraphics[width=\textwidth]{Progetto_2/img/gui_compressed14.png}
        \caption{Immagine compressa F = 8 e d = 14.}
    \end{subfigure}
\end{figure}

\end{document}